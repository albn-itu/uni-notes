\section{Exercise 7}
Let $X$ be the number of times Charlie needs to roll a fair 4-sided die before he sees the outcome 1 the first time. So if Charlie rolls 2, 3, 4, 1 then $X$ is 4.

\subsection{Part a}
\question{
	Compute the mean and variance of $X$.
}

Let $p$ be the probability of rolling a 1 on a 4-sided die. Then $p = \frac{1}{4}$\\

This is a geometric distribution, aka a distribution where we count the number of trials until the first success. The mean and variance of a geometric distribution is given by:

\[
	E[X] = \sum_{k=1}^{\infty}k(1-p)^{k-1}p = \frac{1}{p}
\]

So since we define our problem as $X \sim Geom(p) = Geom\left(\frac{1}{4}\right)$, we have that:
\[
	E[X] = \frac{1}{p} = \frac{1}{\frac{1}{4}} = 4
\]

The variance of a geometric distribution requires us to define $q = 1-p$, so $q = 1-\frac{1}{4} = \frac{3}{4}$. The variance is then given by:
\[
	Var(X) = E[X^2] - E[X]^2
\]

In geometric distributions $E[X^2]$ is given by $\frac{1+q}{p^2}$, so we have:
\[
	E[X^2] = \frac{1+q}{p^2} = \frac{1+\frac{3}{4}}{\left(\frac{1}{4}\right)^2} = \frac{\frac{7}{4}}{\frac{1}{16}} = \frac{7\cdot 16}{4\cdot 1} = 28
\]

So the variance is:
\[
	Var(X) = E[X^2] - E[X]^2 = 28 - 4^2 = 28 - 16 = 12
\]

\subsection{Part b}
\question{
	Compute the mean of $X^2$
}

Done in the previous part, we found that $E[X^2] = 28$.

\subsection{Part c}
\question{
	Charlie also owns a fair 2-sided die (this is just a coin with numbers written on the sides). He now conducts the following experiment: First he rolls the 4-sided die. Let $Z$ be the outcome of that roll. If $Z = 1$ he will roll his 2-sided die until he sees the first 1. If $Z > 1$ he will roll his 4-sided die until he sees a 1. Let $W$ be the number of rolls with the second die, so if $Z = 3$ and he rolls 4, 3, 4, 1, then $W = 4$.\\

	Compute the conditional probability $P(Z = 1 | W = 1)$.
}

This problem adds further 2 geometric distributions, $W_1$ and $W_2$ which denote the number of rolls with the 2-sided die and the 4-sided die respectively:
\begin{align*}
	W_1 & \sim Geom\left(\frac{1}{2}\right) \\
	W_2 & \sim Geom\left(\frac{1}{4}\right)
\end{align*}

Every roll of $Z$ has the probability of $\frac{1}{4}$\\

First we calculate the total probability of $W=1$ using the law of total probability:
\begin{align*}
	P(W=1) & = P(W=1|Z=1)P(Z=1) + P(W=1|Z=2)P(Z=2)                                                               \\
	       & + P(W=1|Z=3)P(Z=3) + P(W=1|Z=4)P(Z=4)                                                               \\
	       & = P(W=1|Z=1)\frac{1}{4} + P(W=1|Z=2)\frac{1}{4} + P(W=1|Z=3)\frac{1}{4} + P(W=1|Z=3)\frac{1}{4}     \\
	       & = W_1\frac{1}{4} + W_2\frac{1}{4} + W_2\frac{1}{4} + W_2\frac{1}{4}                                 \\
	       & = \frac{1}{2}\frac{1}{4} + \frac{1}{4}\frac{1}{4} + \frac{1}{4}\frac{1}{4} + \frac{1}{4}\frac{1}{4} \\
	       & = \frac{1}{8} + \frac{1}{16} + \frac{1}{16} + \frac{1}{16}                                          \\
	       & = \frac{1}{8} + \frac{3}{16} = \frac{2}{16} + \frac{3}{16} = \frac{5}{16}
\end{align*}

Now we can calculate the conditional probability using Bayes theorem
\begin{align*}
	P(Z=1|W=1) & = \frac{P(W=1|Z=1)P(Z=1)}{P(W=1)}                                                                                             \\
	           & = \frac{W_1\frac{1}{4}}{\frac{5}{16}}                                                                                         \\
	           & = \frac{\frac{1}{2}\frac{1}{4}}{\frac{5}{16}} = \frac{\frac{1}{8}}{\frac{5}{16}} = \frac{1}{8}\cdot\frac{16}{5} = \frac{2}{5}
\end{align*}

\subsection{Part d}
\question{
	Compute the mean of $W$
}

The mean of $W$ is given by the total expectation theorem:
\begin{align*}
	E[W]     & = E[W|Z=1]\cdot P(Z=1) + E[W|Z>1]\cdot P(Z>1)                           \\
	E[W|Z=1] & = E[W_1] = \frac{1}{p} = \frac{1}{\frac{1}{2}} = 2                      \\
	E[W|Z>1] & = E[W_2] = \frac{1}{p} = \frac{1}{\frac{1}{4}} = 4                      \\
	E[W]     & = 2\cdot\frac{1}{4} + 4\cdot\frac{3}{4} = \frac{1}{2} + 3 = \frac{7}{2}
\end{align*}

\subsection{Part e}
\question{
	Compute the mean of $W^2$ and the variance of $W$
}

Mean of $W^2$ is given by the total expectation theorem as in the previous part:
\[
	E[W^2]     & = E[W^2|Z=1]\cdot P(Z=1) + E[W^2|Z>1]\cdot P(Z>1)                           \\
\]

We can find $E[W^2|Z=1]$ and $E[W^2|Z>1]$ by:
\[
	E[X^2|Y] = E[X|Y]^2 + Var(X|Y)
\]

We can calculate the variance of $Var(W|Z=1)$ and $Var(W|Z>1)$ by:
\[
	Var(W|Z) = \frac{q}{p^2}
\]

\begin{align*}
	Var(W|Z=1) & = \frac{q}{p^2} = \frac{1-\frac{1}{2}}{\left(\frac{1}{2}\right)^2} = \frac{\frac{1}{2}}{\frac{1}{4}} = 2   \\
	Var(W|Z>1) & = \frac{q}{p^2} = \frac{1-\frac{1}{4}}{\left(\frac{1}{4}\right)^2} = \frac{\frac{3}{4}}{\frac{1}{16}} = 12
\end{align*}


\begin{align*}
	E[W^2|Z=1] & = E[W|Z=1]^2 + Var(W|Z=1) = 2^2 + 2 = 6   \\
	E[W^2|Z>1] & = E[W|Z>1]^2 + Var(W|Z>1) = 4^2 + 12 = 28
\end{align*}

\begin{align*}
	E[W^2] & = 6\cdot\frac{1}{4} + 28\cdot\frac{3}{4}                   \\
	       & = \frac{6}{4} + \frac{84}{4} = \frac{90}{4} = \frac{45}{2}
\end{align*}

The variance of $W$ is given by:
\begin{align*}
	Var(W) & = E[W^2] - E[W]^2                            \\
	       & = \frac{45}{2} - \left(\frac{7}{2}\right)^2  \\
	       & =  \frac{45}{2} - \frac{49}{4}               \\
	       & = \frac{90}{4} - \frac{49}{4} = \frac{41}{4}
\end{align*}
