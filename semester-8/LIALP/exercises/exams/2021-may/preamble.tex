% =======================
% Imports
% =======================
\usepackage[a4paper, bottom=3cm, top=2cm, left=2.6cm, right=2.6cm]{geometry} % layouting
\usepackage{amsmath}
\usepackage{amssymb}
\usepackage{float}
\usepackage{tabularx}
\usepackage{graphicx} % Required for inserting images
\usepackage{tikz}
\usepackage[T1]{fontenc}
\usepackage[utf8]{inputenc}
\usepackage{gauss} % Required for gauss elimination

% =======================
% Layouting
% =======================
\setlength{\parindent}{0pt} % Disable paragraph indentation

% =======================
% Question box
% =======================
\tikzstyle{questionbox} = [
draw=black,
fill=gray!15,
very thick,
rectangle,
rounded corners,
inner sep=10pt,
inner ysep=20pt
]

\newcommand{\question}[1]{%
	\begin{center}
		\begin{tikzpicture}
			\node[questionbox] (box){%
				\begin{minipage}{0.8\textwidth}
					#1
				\end{minipage}
			};
		\end{tikzpicture}
	\end{center}
}


% =======================
% Matrix
% =======================
% Source: https://tex.stackexchange.com/questions/2233/whats-the-best-way-make-an-augmented-coefficient-matrix/2244#2244 
\makeatletter
\renewcommand*\env@matrix[1][*\c@MaxMatrixCols c]{%
	\hskip -\arraycolsep
	\let\@ifnextchar\new@ifnextchar
	\array{#1}}
\makeatother

% source: https://tex.stackexchange.com/questions/241444/augmented-matrix-with-fractions 

% patch gauss macros for doing their work in `align'
% and other amsmath environments; see
% http://tex.stackexchange.com/questions/146532/
\usepackage{etoolbox}
\makeatletter
\patchcmd\g@matrix
{\vbox\bgroup}
{\vbox\bgroup\normalbaselines}% restore the standard baselineskip
{}{}
\makeatother

\newcommand{\gfrac}[2]{\frac{\smash[b]{\mathstrut}#1}{\smash[t]{\mathstrut}#2}}

\newcommand{\BAR}[1][0pt]{%
	\hspace{-\arraycolsep}%
	\ifdim#1>0pt
		\dimen0=\ht\strutbox \dimen2=\dimen0
		\advance\dimen0 #1\relax
		\ht\strutbox=\dimen0
	\fi
	\smash{\strut\vrule} % the `\vrule` is as high and deep as a strut
	% since assignments to \ht\strutbox are global, we restore the height
	\ifdim#1>0pt
		\ht\strutbox=\dimen2
	\fi
	\hspace{-\arraycolsep}%
}

% =======================
% tikz
% =======================
\usetikzlibrary{calc,arrows}
\newcommand{\tikzmark}[1]{%
	\tikz[overlay,remember picture] \node (#1) {};}
