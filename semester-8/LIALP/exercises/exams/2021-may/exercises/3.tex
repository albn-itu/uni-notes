\section{Exercise 3}
For this problem, consider the four vectors in $\mathbb{R}^4$:
\begin{align*}
	v_1 & = \begin{bmatrix}1\\2\\2\\-1\end{bmatrix}
	    & v_2                                       & = \begin{bmatrix}2\\3\\-1\\1\end{bmatrix}
	    & v_3                                       & = \begin{bmatrix}4\\5\\-7\\5\end{bmatrix}
	    & v_4                                       & = \begin{bmatrix}0\\2\\1\\-1\end{bmatrix}
\end{align*}

\subsection{Part a}
\question{
	Is $v_3$ in the subspace $Span({v_1, v_2})$ of $\mathbb{R}^4$? Argue for your answer.
}

To determine if $v_3$ is in the subspace $Span({v_1, v_2})$ we need to determine if $v_3$ can be written as a linear combination of $v_1$ and $v_2$. This is equivalent to solving the system of equations:
\begin{align*}
	\begin{bmatrix}4\\5\\-7\\5\end{bmatrix} & = c_1\begin{bmatrix}1\\2\\2\\-1\end{bmatrix} + c_2\begin{bmatrix}2\\3\\-1\\1\end{bmatrix}
\end{align*}

We can trivially solve this as
\begin{align*}
	 & \begin{gmatrix}[b]
		   1 & 2 & \BAR & 4 \\
		   2 & 3 &\BAR & 5 \\
		   2 & -1 & \BAR & -7 \\
		   -1 & 1 & \BAR & 5
	   \end{gmatrix}\rightsquigarrow        \\
	 & \begin{gmatrix}[b]
		   1 & 2 & \BAR & 4 \\
		   0 & -1 & \BAR & -3 \\
		   0 & -5 & \BAR & -15 \\
		   0 & 3 & \BAR & 9
		   \rowops
		   \add[-2]{0}{1}
		   \add[-2]{0}{2}
		   \add[1]{0}{3}
	   \end{gmatrix}\rightsquigarrow        \\
	 & \begin{gmatrix}[b]
		   1 & 0 & \BAR & -2 \\
		   0 & 0 & \BAR & 0 \\
		   0 & 0 & \BAR & 0 \\
		   0 & 1 & \BAR & 3
		   \rowops
		   \mult{3}{\cdot \frac{1}{3}}
		   \add[5]{3}{2}
		   \add[1]{3}{1}
		   \add[-2]{3}{0}
	   \end{gmatrix}\rightsquigarrow \\
\end{align*}

So we have that $c_1 = -2$ and $c_2 = 3$ which means that $v_3$ is in the subspace $Span({v_1, v_2})$.

\subsection{Part b}
\question{
	Are the vectors $v_1, v_2, v_4$ linearly independent? Argue for your answer.
}

To check for linear independence we need to determine if the equation
\begin{align*}
	c_1\begin{bmatrix}1\\2\\2\\-1\end{bmatrix} + c_2\begin{bmatrix}2\\3\\-1\\1\end{bmatrix} + c_3\begin{bmatrix}0\\2\\1\\-1\end{bmatrix} = \begin{bmatrix}0\\0\\0\\0\end{bmatrix}
\end{align*}

has a non-trivial solution. We can solve this by setting up the augmented matrix and reducing it:

\begin{align*}
	 & \begin{gmatrix}[b]
		   1 & 2 & 0 & \BAR & 0 \\
		   2 & 3 & 2 & \BAR & 0 \\
		   2 & -1 & 1 & \BAR & 0 \\
		   -1 & 1 & -1 & \BAR & 0
	   \end{gmatrix}\rightsquigarrow         \\
	 & \begin{gmatrix}[b]
		   1 & 2 & 0 & \BAR & 0 \\
		   0 & -1 & 2 & \BAR & 0 \\
		   0 & -5 & 1 & \BAR & 0 \\
		   0 & 3 & -1 & \BAR & 0
		   \rowops
		   \add[-2]{0}{1}
		   \add[-2]{0}{2}
		   \add[1]{0}{3}
	   \end{gmatrix}\rightsquigarrow         \\
	 & \begin{gmatrix}[b]
		   1 & 0 & 4 & \BAR & 0 \\
		   0 & 1 & -2 & \BAR & 0 \\
		   0 & 0 & -9 & \BAR & 0 \\
		   0 & 0 & 5 & \BAR & 0
		   \rowops
		   \mult{1}{\cdot -1}
		   \add[2]{1}{0}
		   \add[5]{1}{2}
		   \add[-3]{1}{3}
	   \end{gmatrix}\rightsquigarrow         \\
	 & \begin{gmatrix}[b]
		   1 & 0 & 0 & \BAR & 0 \\
		   0 & 1 & 0 & \BAR & 0 \\
		   0 & 0 & 0 & \BAR & 0 \\
		   0 & 0 & 1 & \BAR & 0
		   \rowops
		   \mult{3}{\cdot \frac{1}{5}}
		   \add[-4]{3}{0}
		   \add[2]{3}{1}
		   \add[9]{3}{2}
	   \end{gmatrix}\rightsquigarrow \\
\end{align*}

From this we can see that the system has a trivial solution, which means that the vectors $v_1, v_2, v_4$ are linearly independent.
