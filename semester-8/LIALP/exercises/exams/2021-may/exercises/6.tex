\section{Exercise 6}
There are 4 pieces of candy in a bowl. Alice walks by and takes 1 piece of candy, and then, with probability $\frac{2}{3}$ she takes one more. Then Bob walks by and picks a number of pieces of candy uniformly distributed between 1 and the number of pieces of candy in the bowl. That is, Bob picks 1, 2 or 3 pieces if Alice only picked one piece or 1 or 2 if she picked 2 pieces. Let $X$ be the number of pieces of candy that Alice picks, and let $Y$ be the number of pieces Bob picks.

\subsection{Part a}
\question{
	Describe the joint PMF (probability mass function) $p_{X,Y}$ for $X$ and $Y$ , i.e., give the value of $p_{X,Y}(k, l)$ for all values of $k$ and $l$
}

$k$ is the amount Alice took, and $l$ is the amount Bob took.

If Alice takes 1 piece, then $p_{X,Y}(1, l)$ is the probability that Bob takes $l$ pieces. This is uniformly distributed between 1 and 3, so it's the chance Alice doesn't take 2 pieces times the chance Bob takes $l$ pieces. This is $\frac{1}{3} \cdot \frac{1}{3} = \frac{1}{9}$ for $l = 1, 2, 3$.\\

If Alice takes 2 pieces, then $p_{X,Y}(2, l)$ is the probability that Bob takes $l$ pieces. This is uniformly distributed between 1 and 2, so it's the chance Alice takes 2 pieces times the chance Bob takes $l$ pieces. This is $\frac{2}{3} \cdot \frac{1}{2} = \frac{1}{3}$ for $l = 1, 2$.\\

In table form:

\begin{center}
	\begin{tabular}{c|cc}
		$P_{X,Y}(k,l)$ & $k = 1$ & $k = 2$ \\ \hline
		$l = 1$        & 1/9     & 1/3     \\
		$l = 2$        & 1/9     & 1/3     \\
		$l = 3$        & 1/9     & 0       \\
	\end{tabular}
\end{center}

\subsection{Part b}
\question{
	Compute the mean of $X$. Compute the conditional means $E[Y | X = 1]$ and $E[Y | X = 2]$ of Y and use these to compute the mean of $Y$.
}

The mean of $X$ is the expected value of the number of pieces Alice takes. Denote as:
\begin{align*}
	S & \sim Ber\left(\frac{2}{3}\right) \\
	X & = 1 + S
\end{align*}

Then the mean of $X$ is:
\begin{align*}
	E[X] & = E[1 + S]        \\
	     & = 1 + E[S]        \\
	     & = 1 + \frac{2}{3} \\
	     & = \frac{5}{3}
\end{align*}

The conditional mean of $Y$ given $X = 1$ is the expected value of the number of pieces Bob takes given that Alice took 1 piece. $Y$ is uniformly distributed between 1 and 3, so the expected value is the average of these values:
\begin{align*}
	E[Y | X = 1] & = \frac{1 + 3}{2} \\
	             & = 2
\end{align*}

The conditional mean of $Y$ given $X = 2$ is the expected value of the number of pieces Bob takes given that Alice took 2 pieces. $Y$ is uniformly distributed between 1 and 2, so the expected value is the average of these values:
\begin{align*}
	E[Y | X = 2] & = \frac{1 + 2}{2} \\
	             & = \frac{3}{2}
\end{align*}

The mean of $Y$ is then the expected value of the number of pieces Bob takes:
\begin{align*}
	E[Y] & = E[E[Y|X]]                                           \\
	     & = 2 \cdot \frac{1}{3} + \frac{3}{2} \cdot \frac{2}{3} \\
	     & = \frac{2}{3} + \frac{6}{6}                           \\
	     & = \frac{4}{6} + \frac{6}{6}                           \\
	     & = \frac{10}{6}                                        \\
	     & = \frac{5}{3}
\end{align*}

\subsection{Part c}
\question{
	Compute the covariance of $X$ and $Y$.
}

The covariance of $X$ and $Y$ is the expected value of the product of the deviations of $X$ and $Y$ from their means:
\begin{align*}
	Cov(X, Y) & = E[XY]-E[X]E[Y]
\end{align*}

We can find $E[XY]$ by:
\[
	E[XY] = \sum_{x\in\mathcal{X}}\sum_{y\in\mathcal{Y}}xyP_{X,Y}(x,y)
\]

\begin{align*}
	E[XY] & = 1 \cdot 1 \cdot \frac{1}{9} + 1 \cdot 2 \cdot \frac{1}{9} + 1 \cdot 3 \cdot \frac{1}{9} + 2 \cdot 1 \cdot \frac{1}{3} + 2 \cdot 2 \cdot \frac{1}{3} \\
	      & = \frac{1}{9} + \frac{2}{9} + \frac{3}{9} + \frac{2}{3} + \frac{4}{3}                                                                                 \\
	      & = \frac{1}{9} + \frac{2}{9} + \frac{3}{9} + \frac{6}{9} + \frac{12}{9}                                                                                \\
	      & = \frac{24}{9}                                                                                                                                        \\
\end{align*}

We can then calculate the covariance as:
\begin{align*}
	Cov(X, Y) & = E[XY]-E[X]E[Y]                               \\
	          & = \frac{24}{9} - \frac{5}{3} \cdot \frac{5}{3} \\
	          & = \frac{24}{9} - \frac{25}{9}                  \\
	          & = -\frac{1}{9}
\end{align*}

\subsection{Part d}
\question{
	Now, let $Z$ be the total number of pieces of candy picked by Alice and Bob. \\

	Compute the variances of $X$, $Y$ and $Z$.
}

First compute the variance of $X$:
\begin{align*}
	E[X^2] & = 1\cdot \frac{1}{3} + 4\cdot \frac{2}{3} \\
	       & = \frac{1}{3} + \frac{8}{3}               \\
	       & = \frac{9}{3}                             \\
	       & = 3                                       \\
	Var(X) & = E[X^2] - E[X]^2                         \\
	       & = 3 - \left(\frac{5}{3}\right)^2          \\
	       & = 3 - \frac{25}{9}                        \\
	       & = \frac{27}{9} - \frac{25}{9}             \\
	       & = \frac{2}{9}
\end{align*}

Then compute the variance of $Y$:
\begin{align*}
	E[Y^2] & = \frac{1}{9}(1 + 4 + 9) + \frac{1}{3}(1+4)                           \\
	       & = \frac{1}{9} + \frac{4}{9} + \frac{9}{9} + \frac{1}{3} + \frac{4}{3} \\
	       & = \frac{14}{9} + \frac{3}{9} + \frac{12}{9}                           \\
	       & = \frac{14+15}{9} = \frac{29}{9}                                      \\
	Var(Y) & = E[Y^2] - E[Y]^2                                                     \\
	       & = \frac{29}{9} - \left(\frac{5}{3}\right)^2                           \\
	       & = \frac{29}{9} - \frac{25}{9}                                         \\
	       & = \frac{4}{9}                                                         \\
\end{align*}
