\section{Exercise 7}
\question{
	An ice cream shop sells three flavours of ice cream, which we will simply refer to as $a$, $b$, and $c$. Customers can choose between 1,2 and 3 scoops of ice cream in each cup. Alice likes all the three flavours and is totally undecided as she enters the shop. She decides to make a uniform choice of cup size (1,2 or 3 scoops) and after that, a uniform choice of flavours of ice cream for the cup, making sure that she will choose at most one scoop of each flavour. In other words, if she chooses a cup for 2 scoops, then any of the three flavour combinations $ab$, $ac$ and $bc$ are equally likely. Let $A$, $B$, $C$ denote the events that flavours $a$,$b$ and $c$, respectively appear in the cup Alice buys. Let $X_2$ be the event that she chooses a cup for two scoops.
}

\subsection{Part a}
\question{
	Are $A$ and $B$ independent?
}

First we can conclude that:
\[
	P(A) = P(B) = P(C) = \frac{1}{3}
\]

Then we can conclude that:
\begin{align*}
	P(A|X_1) & = \frac{1}{3} \cdot \frac{1}{3} = \frac{1}{9} \\
	P(A|X_2) & = \frac{1}{3} \cdot \frac{2}{3} = \frac{2}{9} \\
	P(A|X_3) & = \frac{1}{3} \cdot \frac{3}{3} = \frac{1}{3}
\end{align*}

These count for $B$ and $C$ too.

Then we can conclude that:
\[
	P(A) = \frac{1}{9} + \frac{2}{9} + \frac{1}{3} = \frac{1}{9} + \frac{2}{9} + \frac{3}{9} = \frac{6}{9} = \frac{2}{3}
\]

This is the same for $B$ and $C$.

Then we can calculate $P(A \cap B)$:
\begin{align*}
	P(A\cap B|X_1) & = \frac{1}{3} \cdot 0 = 0                                                 \\
	P(A\cap B|X_2) & = \frac{1}{3} \cdot \frac{1}{3} = \frac{1}{9}                             \\
	P(A\cap B|X_3) & = \frac{1}{3} \cdot \frac{3}{3} = \frac{1}{3}                             \\
	P(A \cap B)    & = 0 + \frac{1}{9} + \frac{1}{3} = \frac{1}{9} + \frac{3}{9} = \frac{4}{9}
\end{align*}

We can then check independence by checking if $P(A \cap B) = P(A)P(B)$:
\[
	\frac{4}{9} = \frac{2}{3} \cdot \frac{2}{3} = \frac{4}{9}
\]

So $A$ and $B$ are independent

\section{Part b}
\question{
	Are $A$ and $B$ independent conditional on $X_2$? Argue mathematically for your answer.
}

We already have $P(A|X_2) = P(B|X_2) = \frac{2}{9}$ so we must calculate $P(A|X_2 \cap B|X_2) = P(A\cap B|X_2)$. Which we actually also have, so:
\begin{align*}
	\frac{1}{9} & = \frac{2}{9} \cdot \frac{2}{9} \\
	\frac{1}{9} & = \frac{4}{9}                   \\
\end{align*}

So $A$ and $B$ are not independent conditional on $X_2$.

\section{Part c}
\question{
	Later that summer on a sunny day, the ice cream shop is all out of flavours $b$ and $c$, and only sells flavour $a$. Despite that being clearly marked at the shop, 16 customers are waiting in line. Experience shows that each customer will want 2 scoops of ice cream a with probability $\frac{1}{2}$ and 3 scoops with probability $\frac{1}{2}$. No customers will choose just one scoop. There is only ice cream left for 37 scoops.\\[1ex]

	Use the normal approximation of the binomial to estimate the probability that there is enough ice cream for all 16 customers in the queue.
}

Let $S$ be the number of scoops of ice cream sold. Then $S_i$ is the number of scoops sold to customer $i$. This can be modelled as a boolean of whether the customer wants 2 or 3 scoops.

We can then calculate the mean and variance of $S_i$:
\begin{align*}
	E[S_i]   & = 2\frac{1}{2} + 3\frac{1}{2} = 2.5              \\
	Var(S_i) & = 2^2\frac{1}{2} + 3^2\frac{1}{2} - 2.5^2 = 0.25
\end{align*}

Then we can define the value $S$ as
\[
	S = \sum^16_{i=1} S_i
\]

And the mean and variance of $S$ is:
\begin{align*}
	E[S]   & = 16 \cdot 2.5 = 40 \\
	Var(S) & = 16 \cdot 0.25 = 4
\end{align*}

To calculate the probability we must use the central limit theorem. It formulates as
\[
	Z_n = \frac{S_n - E[S_n]}{\sqrt{Var(S_n)}}
\]

We want to calculate $P(S\leq 37)$ so $S_n$ is 37. We can then calculate $Z_n$:
\[
	Z_n = \frac{37 - 40}{\sqrt{4}} = \frac{-3}{2} = -1.5
\]

This leaves us with the probability $P(Z_n \leq -1.5) = \Phi(-1.5)$. We can then look up the value in a normal distribution table, which is $0.0668$. So the probability that there is enough ice cream for all 16 customers in the queue is $0.0668$.

\section{Part d}
\question{
	Even later that summer, the ice cream shop sells 5 different flavours, which we refer to as $a$,$b$,$c$,$d$,$e$. Bob wants a cup with 2 scoops, and will choose uniformly among the flavours, making sure not to choose the same flavour twice. Let $Y_A$ and $Y_B$ be the Bernoulli random variables associated with the events that Bob’s ice cream contains flavours $a$ and $b$ respectively.\\[1ex]

	Compute the mean and variance of $Y_A$, and the covariance of $Y_A$ and $Y_B$ .
}

We define $Y_A$ and $Y_B$ as:
\begin{align*}
	Y_A & \sim \text{Bernoulli}(p) \\
	Y_B & \sim \text{Bernoulli}(p)
\end{align*}

There is a total of $\begin{pmatrix}5\\2\end{pmatrix} = 10$ possible combinations of flavours. Now, there are 4 combinations of flavours that contain $a$ and 4 combinations of flavours that contain $b$. So $p = \frac{4}{10} = \frac{2}{5}$.

Combinations:
\[
	\{ab, ac, ad, ae, bc, bd, be, cd, ce, de\}
\]

The mean and variance of a Bernoulli distribution is:
\begin{align*}
	E[Y]   & = p      \\
	Var(Y) & = p(1-p)
\end{align*}

So the mean and variance of $Y_A$ is:
\begin{align*}
	E[Y_A]   & = \frac{2}{5}                                  \\
	Var(Y_A) & = \frac{2}{5} \cdot \frac{3}{5} = \frac{6}{25}
\end{align*}

The covariance of $Y_A$ and $Y_B$ is:
\[
	Cov(Y_A, Y_B) = E[Y_AY_B] - E[Y_A]E[Y_B]
\]

We must calculate $E[Y_AY_B]$, which is the probability that both $a$ and $b$ are present. There is only one case where those sets overlap, so it's:
\begin{align*}
	Y_AY_B    & \sim \text{Bernoulli}(p^2) \\
	p         & = \frac{1}{10}             \\
	E[Y_AY_B] & = \frac{1}{10}
\end{align*}

So the covariance of $Y_A$ and $Y_B$ is:
\begin{align*}
	Cov(Y_A, Y_B) & = \frac{1}{10} - \frac{2}{5} \cdot \frac{2}{5} \\
	              & = \frac{1}{10} - \frac{4}{25}                  \\
	              & = \frac{5}{50} - \frac{8}{50} = -\frac{3}{50}
\end{align*}

\section{Part e}
\question{
	Suppose now that the ice cream shop charges per scoop as given in the table below. So for example, an ice cream with one scoop of flavour $a$ and one scoop of flavour $b$ is 5 kr. Let $Z$ be the price of Bob’s ice cream\\[1ex]

	\begin{center}
		\begin{tabular}{|c|c|c|c|c|c|}
			\hline
			flavour & a & b & c & d & e \\
			price   & 2 & 3 & 4 & 4 & 5 \\
			\hline
		\end{tabular}
	\end{center}

	Compute the mean of $Z$ and the covariance of $Y_A$ and $Z$. (Hint: use the results of the previous question)
}

We define $Z$ as:
\[
	Z = 2Y_A + 3Y_B + 4Y_C + 4Y_D + 5Y_E
\]

The mean of $Z$ is:
\begin{align*}
	E[Z] & = E[2Y_A + 3Y_B + 4Y_C + 4Y_D + 5Y_E]                                      \\
	     & = 2E[Y_A] + 3E[Y_B] + 4E[Y_C] + 4E[Y_D] + 5E[Y_E]                          \\
	     & = 2\frac{2}{5} + 3\frac{2}{5} + 4\frac{2}{5} + 4\frac{2}{5} + 5\frac{2}{5} \\
	     & = \frac{2}{5} + \frac{6}{5} + \frac{8}{5} + \frac{8}{5} + \frac{10}{5}     \\
	     & = \frac{10}{5} + \frac{16}{5} + \frac{10}{5} = \frac{36}{5}
\end{align*}

The covariance of $Y_A$ and $Z$ is:
\[
	Cov(Y_A, Z) = E[Y_AZ] - E[Y_A]E[Z]
\]

Since the covariance consists of variables, a solution is to calculate it as so:
\[
	Cov(Y_A, Z) = 2\cdot Cov(Y_A, Y_A) + 3\cdot Cov(Y_A, Y_B) + 4\cdot Cov(Y_A, Y_C) + 4\cdot Cov(Y_A, Y_D) + 5\cdot Cov(Y_A, Y_E)
\]

We know all apart from $Cov(Y_A, Y_A)$ which, when getting the covariance of the same variable, is just the variance of the variable. Therefore we have:
\begin{align*}
	Cov(Y_A, Z) & = 2\cdot \frac{6}{25} + (-\frac{3}{50}(3+4+5+6)) \\
	            & = \frac{12}{25} - \frac{3}{50} \cdot 16          \\
	            & = \frac{24}{50} - \frac{48}{50}                  \\
	            & = \frac{-24}{50}
\end{align*}



