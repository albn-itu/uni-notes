\section{Exercise 7}
Anna looks forward to hearing the new mini-album by her favourite artist. The album contains 5 songs out of which 2 are hits and 3 turn out to be boring songs. Anna plays the album on shuffle, which means that the 5 songs will be played in random order without repeat. We will assume that all possible orderings of the 5 songs are equally likely.\\

Let $X$ be a random variable indicating the number of songs played when Anna has heard the first hit. For example, if the two first songs played are boring, but the third is a hit, then $X = 3$

\subsection{Part a}
\question{
	Describe the PMF (probability mass function) for $X$.
}

The PMF would look like:
\[
	p(k)=P(X=k)
\]

Where $k$ is the first hit song. Each value looks like:
\begin{align*}
	p_X(1) & = \frac{2}{5}  &
	p_X(2) & = \frac{3}{10} &
	p_X(3) & = \frac{1}{5}  &
	p_X(4) & = \frac{1}{10}
\end{align*}

\subsection{Part b}
\question{
	Compute the mean of $X$, the mean of $X^2$ and the variance of $X$.
}

The expected value of $X$ is:
\[
	E[X] = \sum_{k=1}^{4} k \cdot p_X(X=k)
\]

\begin{align*}
	E[X] & = 1 \cdot \frac{2}{5} + 2 \cdot \frac{3}{10} + 3 \cdot \frac{1}{5} + 4 \cdot \frac{1}{10} \\
	     & = \frac{2}{5} + \frac{6}{10} + \frac{3}{5} + \frac{4}{10}                                 \\
	     & = \frac{2}{5} + \frac{3}{5} + \frac{3}{5} + \frac{2}{5}                                   \\
	     & = 2
\end{align*}

Compute $E[X^2]$:
\[
	E[X^2] = \sum_{k=1}^{4} k \cdot p_X(X^2=k)
\]

\begin{align*}
	E[X^2] & = 1 \cdot \frac{2}{5} + 4 \cdot \frac{3}{10} + 9 \cdot \frac{1}{5} + 16 \cdot \frac{1}{10} \\
	       & = \frac{2}{5} + \frac{12}{10} + \frac{9}{5} + \frac{16}{10}                                \\
	       & = \frac{2}{5} + \frac{6}{5} + \frac{9}{5} + \frac{8}{5}                                    \\
	       & = 5
\end{align*}

The variance of $X$ is:
\[
	Var(X) = E[X^2] - E[X]^2
\]

\begin{align*}
	Var(X) & = 5 - 2^2 \\
	       & = 5 - 4   \\
	       & = 1
\end{align*}

\subsection{Part c}
\question{
	Now let $B_1$ be the event that the first song played is boring, and let $Y$ be the random variable indicating the number of songs played before Anna has heard both a hit and a boring song. For example, if all three boring songs are played before any of the hits, then $Y = 4$\\

	Describe the conditional PMF of $Y$, given $B_1$ and compute the conditional mean $E[Y | B_1]$
}

The conditional PMF of $Y$ is:
\begin{align*}
	p_{Y|B_1}(2) & = \frac{1}{2}                                &
	p_{Y|B_1}(3) & = \frac{1}{2}\cdot \frac{2}{3} = \frac{1}{3} &
	p_{Y|B_1}(4) & = \frac{1}{6}                                &
\end{align*}

Compute $E[Y|B_1]$:
\[
	E[Y|B_1] = \sum_{k=2}^{4} k \cdot p_{Y|B_1}(Y=k)
\]

\begin{align*}
	E[Y|B_1] & = 2 \cdot \frac{1}{2} + 3 \cdot \frac{1}{3} + 4 \cdot \frac{1}{6} \\
	         & = 1 + 1 + \frac{2}{3}                                             \\
	         & = 2\frac{2}{3} = \frac{8}{3}
\end{align*}

\subsection{Part d}
\question{
	Compute the conditional probability $P(B_1 | Y = 3)$
}

Calculate $P(B_1|Y=3)$:
\[
	P(B_1|Y=3) = \frac{P(Y=3|B_1)P(B_1)}{P(Y=3)}
\]

Let $B^C$ be the event that the first song is a hit:
\[
	P(Y=3|B_1) = \frac{1}{2}\cdot\frac{2}{3} = \frac{1}{3}
\]

\[
	P(Y=3|B^C) = \frac{3}{4}\cdot \frac{1}{3} = \frac{1}{4}
\]

\begin{align*}
	P(Y=3) & = P(Y=3|B_1)P(B_1) + P(Y=3|B^C)P(B^C)                           \\
	       & = \frac{1}{3} \cdot \frac{3}{5} + \frac{1}{4} \cdot \frac{2}{5} \\
	       & = \frac{1}{5} + \frac{1}{10}                                    \\
	       & = \frac{3}{10}
\end{align*}

\begin{align*}
	P(B_1|Y=3) & = \frac{\frac{1}{3} \cdot \frac{3}{5}}{\frac{3}{10}} \\
	           & = \frac{2}{3}
\end{align*}
