\section{Exercise 1}
\question{
	Let $v$ be the vector
	$$
		v = \begin{bmatrix}
			1 \\
			2
		\end{bmatrix}
	$$
	and consider the subset
	$$
		V = \{ A\in M_{2,2} | v^TA = (Av)^T \}
	$$
	of the vector space $M_{2,2}$ of $2 \times 2$ matrices. Here $v^T$ is the transpose of $v$. Consider also the matrix
	$$
		B = \begin{bmatrix}
			1 & 3 \\
			3 & 1
		\end{bmatrix}
	$$
	Answer the following questions:
	\begin{description}
		\item (a) Is the matrix B an element in V ?
		\item (b) Is V a subspace of $M_{2,2}$?
	\end{description}
	In both cases you should argue for your answer. You may use the following equalities
	$$
		(D + E)^T = D^T + E^T \quad (cD)^T = c(D^T)
	$$
	You do not have to argue for these or prove them. These equalities hold for all matrices $D$, $E$ and scalars $c$ whenever $D$ and $E$ have the same dimensions
}

\subsection{Part a}
We want to determine if the matrix $B$ is an element in $V$. We do this by checking if $v^TB = (Bv)^T$.

First we calculate $v^TB$:
\begin{align*}
	v^TB & = \begin{bmatrix}
		         1 & 2
	         \end{bmatrix}
	\begin{bmatrix}
		1 & 3 \\
		3 & 1
	\end{bmatrix}                                         \\
	     & = \begin{bmatrix}
		         1 \cdot 1 + 2 \cdot 3 & 1 \cdot 3 + 2 \cdot 1
	         \end{bmatrix} \\
	     & = \begin{bmatrix}
		         7 & 5
	         \end{bmatrix}
\end{align*}

Then we calculate $(Bv)^T$:
\begin{align*}
	(Bv)^T & = \left(\begin{bmatrix}
		                 1 & 3 \\
		                 3 & 1
	                 \end{bmatrix}
	\begin{bmatrix}
		1 \\
		2
	\end{bmatrix}\right)^T                  \\
	       & = \left(\begin{bmatrix}
		                 1 \cdot 1 + 3 \cdot 2 \\
		                 3 \cdot 1 + 1 \cdot 2
	                 \end{bmatrix}\right)^T \\
	       & = \begin{bmatrix}
		           7 \\
		           5
	           \end{bmatrix}^T              \\
	       & = \begin{bmatrix}
		           7 & 5
	           \end{bmatrix}
\end{align*}

Therefore we have that $v^TB = (Bv)^T$, and thus $B$ is an element in $V$.

\subsection{Part b}
We want to determine if $V$ is a subspace of $M_{2,2}$. We do this by checking if $V$ is closed under addition and scalar multiplication.

\subsubsection{Closed under addition}
We want to determine if $V$ is closed under addition. We do this by checking if $A, B \in V$ implies that $A + B \in V$.

Let $A, B \in V$. Then we have that $v^TA = (Av)^T$ and $v^TB = (Bv)^T$. We then check if $v^T(A + B) = ((A + B)v)^T$.

First we calculate $v^T(A + B)$:
\begin{align*}
	v^T(A + B) & = v^TA + v^TB     \\
	           & = (Av)^T + (Bv)^T
\end{align*}

Then we calculate $((A + B)v)^T$:
\begin{align*}
	((A + B)v)^T & = (Av + Bv)^T     \\
	             & = (Av)^T + (Bv)^T
\end{align*}

Therefore we have that $v^T(A + B) = ((A + B)v)^T$, and thus $A + B \in V$. This means that $V$ is closed under addition.

\subsubsection{Closed under scalar multiplication}
We want to determine if $V$ is closed under scalar multiplication. We do this by checking if $A \in V$ implies that $cA \in V$ for all scalars $c$.

Let $A \in V$. Then we have that $v^TA = (Av)^T$. We then check if $v^T(cA) = ((cA)v)^T$.

First we calculate $v^T(cA)$:
\begin{align*}
	v^T(cA) & = cv^TA   \\
	        & = c(Av)^T
\end{align*}

Then we calculate $((cA)v)^T$:
\begin{align*}
	((cA)v)^T & = (cA)^Tv^T \\
	          & = cA^Tv^T   \\
	          & = c(Av)^T
\end{align*}

Therefore we have that $v^T(cA) = ((cA)v)^T$, and thus $cA \in V$. This means that $V$ is closed under scalar multiplication.

