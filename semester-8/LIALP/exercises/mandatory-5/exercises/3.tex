\section{Exercise 3}
\question{
	A man walks in the woods with his two dogs and gets lost. He comes to a fork in the road and knows that each dog will choose the fork that takes him home with probability $p$ independently of the other. He decides that if both dogs choose the same direction he will follow that. Otherwise, he will choose a fork at random. What is his probability of choosing the right fork with this strategy? How does it compare to the strategy of picking one dog in advance, and following that? Explain your reasoning using the laws of probability (especially those concerning conditional probability) as carefully as you can.
}

The event $A$ that dog 1 chooses the right fork, and the event $B$ that dog 2 chooses the right fork. The probability of each being $p$. The probability of both dogs choosing the right fork, due to independence is:
\[
	P(A \cap B) = P(A) \cdot P(B) = p^2
\]

The probability of the dogs choosing different forks has the probability:
\begin{align*}
	P((A\cap B)^c) & = P(A) \cdot P(B^c) + P(A^c) \cdot P(B) \\
	               & = p(1-p) + (1-p)p                       \\
	               & = 2p(1-p)
\end{align*}

Lastly we can conclude the probability of the hunter choosing the right fork when the dogs disagree is $0.5$ since he chooses at random between 2 outcomes.\\[1ex]

The probability of choosing the right fork with this strategy is then:
\begin{align*}
	P(\text{Right fork}) & = P(\text{Both dogs choose right}) + P(\text{Dogs disagree}) \cdot P(\text{Hunter chooses right}) \\
	                     & = P(A \cap B) + P((A\cap B)^c) \cdot 0.5                                                          \\
	                     & = p^2 + 0.5 \cdot 2p(1-p)                                                                         \\
	                     & = p^2 + p(1-p)                                                                                    \\
	                     & = p^2 + p - p^2                                                                                   \\
	                     & = p
\end{align*}

So the probability of choosing the right fork with this strategy is $p$.\\[1ex]

Therefore there is no theoretical difference between this strategy and picking a dog in advance and following that, since each dog has a probability of $p$ of choosing the right fork. The probability of choosing the right fork is $p$ in both cases.\\[1em]
