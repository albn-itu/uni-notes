\section{Exercise 2}
\question{
	Suppose we are given an urn containing 4 balls marked by the numbers 1 through 4. Consider the experiment of drawing two balls without replacement from the urn. Define a probability model for this situation in such a way that one can define a random variable $X$ for the minimum number of the two balls drawn and a random variable $Y$ for the maximum. Define $X$ and $Y$, describe the sets $X^{-1}(k)$ for $k = 1, 2, 3, 4$ and use this to compute the PMF of $X$.
}

First we define the sample space of the experiment as
\[
	\Omega = \{ \{a, b\} \subseteq \{1, 2, 3, 4\} \mid a \neq b \}
\]

The size of the sample space, which is unordered without replacement, then is
\[
	\# \Omega = \binom{4}{2} = \frac{4!}{2!2!} = \frac{24}{4} = 6
\]

Next, we define the random variables $X$ and $Y$. The random variable $X$ is the minimum of the two balls drawn, and the random variable $Y$ is the maximum of the two balls drawn. We can define these random variables as
\[
	X(\{a, b\}) = \min(a, b) \quad \text{and} \quad Y(\{a, b\}) = \max(a, b)
\]

We can then define the sets $X^{-1}(k)$ for $k = 1, 2, 3, 4$ as
\begin{align*}
	X^{-1}(1) & = \{ \{1, 2\}, \{1, 3\}, \{1, 4\} \} \\
	X^{-1}(2) & = \{ \{2, 3\}, \{2, 4\} \}           \\
	X^{-1}(3) & = \{ \{3, 4\} \}                     \\
	X^{-1}(4) & = \emptyset
\end{align*}

We can use these sets to compute the PMF of $X$. The PMF of $X$ is given by
\[
	P_X(k) = P(X = k) = P(X^{-1}(k))
\]

So we can calculate the PMF of $X$ as
\begin{align*}
	P_X(1) & = P(X^{-1}(1)) = \frac{\# X^{-1}(1)}{\# \Omega} = \frac{3}{6} = \frac{1}{2} = 0.50 = 50\%            \\
	P_X(2) & = P(X^{-1}(2)) = \frac{\# X^{-1}(2)}{\# \Omega} = \frac{2}{6} = \frac{1}{3} \approx 0.3333 = 33.33\% \\
	P_X(3) & = P(X^{-1}(3)) = \frac{\# X^{-1}(3)}{\# \Omega} = \frac{1}{6} \approx 0.1666 = 16.66\%               \\
	P_X(4) & = P(X^{-1}(4)) = \frac{\# X^{-1}(4)}{\# \Omega} = \frac{0}{6} = 0 = 0\%
\end{align*}

