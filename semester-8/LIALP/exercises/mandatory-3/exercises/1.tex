\section{Exercise 1}
\question{
	Consider the number sequence given by
	\begin{align*}
		x_0     & = 0                                                      \\
		x_1     & = 1                                                      \\
		x_{n+2} & = 3\cdot x_{n+1} - 2\cdot x_n \quad \text{for } n \geq 0
	\end{align*}

	Construct a matrix $A$ such that
	\[
		\begin{bmatrix}
			x_{n+2} \\
			x_{n+1}
		\end{bmatrix}
		= A
		\begin{bmatrix}
			x_{n+1} \\
			x_{n}
		\end{bmatrix}
	\]

	Check that your $A$ is correct by verifying that
	\[
		A
		\begin{bmatrix}
			1 \\
			0
		\end{bmatrix}
		=
		\begin{bmatrix}
			3 \\
			1
		\end{bmatrix}
	\]

	Compute the eigenvectors and eigenvalues of A. Write the vector $\begin{bmatrix}1\\0\end{bmatrix}$ as a linear combination of eigenvectors for $A$ and use this to derive a closed formula for $x_n$.
}

First we derive from the given sequence that
\begin{align*}
	x_{n+2} & = 3\cdot x_{n+1} - 2\cdot x_n \\
	x_{n+1} & = x_{n+1}
\end{align*}

We can rewrite this set of equations as
\[
	\begin{bmatrix}
		x_{n+2} \\
		x_{n+1}
	\end{bmatrix} =
	\begin{bmatrix}
		3 & -2 \\
		1 & 0
	\end{bmatrix}
	\begin{bmatrix}
		x_{n+1} \\
		x_{n}
	\end{bmatrix}
\]

Thereby we derive the matrix $A$ as
\[
	A =
	\begin{bmatrix}
		3 & -2 \\
		1 & 0
	\end{bmatrix}
\]

We can verify that this is correct by checking that
\begin{align*}
	A
	\begin{bmatrix}
		1 \\
		0
	\end{bmatrix}
	=
	\begin{bmatrix}
		3 \\
		1
	\end{bmatrix}
\end{align*}
\begin{align*}
	A
	\begin{bmatrix}
		1 \\
		0
	\end{bmatrix}
	 & =
	\begin{bmatrix}
		3 & -2 \\
		1 & 0
	\end{bmatrix}
	\begin{bmatrix}
		1 \\
		0
	\end{bmatrix}         \\
	 & =
	\begin{bmatrix}
		3\cdot 1 + (-2)\cdot 0 \\
		1\cdot 1 + 0\cdot 0
	\end{bmatrix} \\
	 & =
	\begin{bmatrix}
		3 \\
		1
	\end{bmatrix}
\end{align*}

Thereby we have verified that $A$ is correct.\\[2ex]
We can now compute the eigenvalues and eigenvectors of $A$. We start by finding the eigenvalues by solving the characteristic polynomial:

\begin{align*}
	|A-\lambda I| & = \begin{vmatrix}
		                  3-\lambda & -2        \\
		                  1         & 0-\lambda
	                  \end{vmatrix}                      \\
	              & = det\left(\begin{vmatrix}
			                           3-\lambda & -2       \\
			                           1         & -\lambda
		                           \end{vmatrix}\right)              \\
	              & = (3 - \lambda) \cdot (-\lambda) - 2 \cdot 1 \\
	              & = \lambda^2 - 3\lambda + 2
	\intertext{Resulting in}
	              & = (\lambda - 2)(\lambda - 1)
\end{align*}

Thereby the characteristic equation is $(\lambda - 1)(\lambda - 2) = 0$ which gives the eigenvalues $\lambda_1 = 2$ and $\lambda_2 = 1$ of $A$.\\[2ex]
To find the corresponding eigenvectors we solve the homogenous linear system $(A-\lambda I)\mathbf{x}=\mathbf{0}$ for each eigenvalue. First for $\lambda_1 = 2$:
\begin{align*}
	2I - A & = \begin{bmatrix}
		           2 - 3 & 2     \\
		           -1    & 2 - 0
	           \end{bmatrix}                \\
	       & = \begin{bmatrix}
		           -1 & 2 \\
		           -1 & 2
	           \end{bmatrix}                \\
	\intertext{Which row reduces to}
	       & \rightsquigarrow \begin{bmatrix}
		                          1 & -2 \\
		                          0 & 0
	                          \end{bmatrix}
\end{align*}

Thus we have that $x_1 - 2x_2 = 0$, with $x_2=t$ we can conclude that every eigenvector $\lambda_1$ is of the form
\[
	\mathbf{x} = \begin{bmatrix}x_1\\x_2\end{bmatrix} = \begin{bmatrix}2t\\t\end{bmatrix} = t\begin{bmatrix}2\\1\end{bmatrix}, \quad t\neq 0
\]

Now for $\lambda_2 = 1$:
\begin{align*}
	1I - A & = \begin{bmatrix}
		           1 - 3 & 2     \\
		           -1    & 1 - 0
	           \end{bmatrix}               \\
	       & = \begin{bmatrix}
		           -2 & 2 \\
		           -1 & 1
	           \end{bmatrix}               \\
	\intertext{Which row reduces to}
	       & \rightsquigarrow\begin{bmatrix}
		                         1 & -1 \\
		                         0 & 0
	                         \end{bmatrix}
\end{align*}

Thus we have that $x_1 - x_2 = 0$, with $x_2=t$ we can conclude that every eigenvector $\lambda_2$ is of the form
\[
	\mathbf{x} = \begin{bmatrix}x_1\\x_2\end{bmatrix} = \begin{bmatrix}t\\t\end{bmatrix} = t\begin{bmatrix}1\\1\end{bmatrix}, \quad t\neq 0
\]

So the eigenvectors for $A$ are $\mathbf{v_1} = \begin{bmatrix}2\\1\end{bmatrix}$ and $\mathbf{v_2} = \begin{bmatrix}1\\1\end{bmatrix}$.\\[2ex]
Now, we can write the vector $\begin{bmatrix}1\\0\end{bmatrix}$ as a linear combination of the eigenvectors for $A$:
\[
	\begin{bmatrix}1\\0\end{bmatrix} = c_1\begin{bmatrix}2\\1\end{bmatrix} + c_2\begin{bmatrix}1\\1\end{bmatrix}
\]

We can solve for $c_1$ and $c_2$ by setting up the following system of equations:
\begin{align*}
	2c_1 + c_2 & = 1 \\
	c_1 + c_2  & = 0
\end{align*}

Which results in $c_1 = 1$ and $c_2 = -1$. So
\[
	\begin{bmatrix}1\\0\end{bmatrix} = 1\begin{bmatrix}2\\1\end{bmatrix} - 1\begin{bmatrix}1\\1\end{bmatrix}
\]

We can express this as $w = \mathbf{v_1} - \mathbf{v_2}$.\\[2ex]
\begin{align*}
	\intertext{If we were to re-express our earlier calculation we would have}
	\begin{bmatrix}x_{n+1}\\x_n\end{bmatrix} & =  A^n\begin{bmatrix}1\\0\end{bmatrix}                                                                  \\
	                                         & = A^n\cdot w                                                                                            \\
	                                         & = A^n\cdot (\mathbf{v_1}-\mathbf{v_2})                                                                  \\
	                                         & = A^n\cdot \mathbf{v_1}- A^n \cdot \mathbf{v_2}                                                         \\
	                                         & = \lambda_1^n\cdot \mathbf{v_1}- \lambda_2^n \cdot \mathbf{v_2}                                         \\
	\intertext{If we replace $\mathbf{v_1}$ and $\mathbf{v_2}$ with their respective eigenvectors we get}
	                                         & = \lambda_1^n\cdot \begin{bmatrix}2\\1\end{bmatrix}- \lambda_2^n \cdot \begin{bmatrix}1\\1\end{bmatrix} \\
	                                         & = \begin{bmatrix}\lambda_1^{n}\cdot2-\lambda_2^n\\\lambda_1^n-\lambda_2^n\end{bmatrix}
\end{align*}

So we have derived a closed formula for $x_n$:
\[
	x_n = \lambda_1^{n}-\lambda_2^n
\]

If we replace $\lambda_1$ and $\lambda_2$ with their respective eigenvalues we get
\[
	x_n = 2^n-1^n = 2^n-1
\]

So the closed formula for $x_n$ is
\[
	x_n = 2^n-1
\]

Verify that this is correct by checking the first few values of $x_n$:
\begin{align*}
	x_0 & = 2^0-1 = 1-1 = 0 \\
	x_1 & = 2^1-1 = 2-1 = 1 \\
	x_2 & = 2^2-1 = 4-1 = 3 \\
	x_3 & = 2^3-1 = 8-1 = 7
\end{align*}
