\section{Exercise 3}
\question{
	Suppose I roll a fair 4-sided die 4 times. What is the probability of seeing at least one 4? (Hint: There is a hard way and an easy way to answer this question.
}

First we define the sample space of the experiment. The sample space is the set of all possible outcomes of the experiment. In this case, the sample space is the set of all possible sequences of 4 rolls of a 4-sided die. Since each roll has 4 possible outcomes, the sample space has $4^4 = 256$ elements. Define it as
\[
	\Omega = \{ (a_1, a_2, a_3, a_4) \mid a_i \in \{1, 2, 3, 4\} \}
\]

Next, we define the event of interest. The event of interest is the set of all outcomes in which at least one 4 is rolled. We can define this event as
\[
	A = \{ (a_1, a_2, a_3, a_4) \in \Omega \mid a_i = 4 \text{ for some } i \}
\]

We want to calculate the probability of this event, which is given by
\[
	P(A) = \frac{\# A}{\#\Omega}
\]

The hard way to solve this problem is to count the number of outcomes in the space and then calculate the probability. The easy way to solve this problem is to use the complement rule.\\[2ex]
The complement rule states that the probability of an event not happening is equal to 1 minus the probability of the event happening. In this case, the event is seeing at least one 4. The complement of this event is seeing no 4s. We can calculate the probability of seeing no 4s and then subtract that from 1 to get the probability of seeing at least one 4. Define the sample space then as
\[
	A^c = \{ (a_1, a_2, a_3, a_4) \in \Omega \mid a_i \neq 4 \text{ for all } i \}
\]

We can calculate the probability of seeing no 4s as
\[
	P(A^c) = \frac{\# A^c}{\#\Omega} = \frac{3^4}{4^4} = \frac{81}{256}
\]

Finally, we can subtract this probability from 1 to get the probability of seeing at least one 4:
\[
	P(A) = 1 - P(A^c) = 1 - \frac{81}{256} = \frac{175}{256} \approx 0.6836 = 68.36\%
\]
