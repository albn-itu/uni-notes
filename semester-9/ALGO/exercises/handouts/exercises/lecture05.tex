\section{Lecture 5}
\subsection{Q1}
\question{
  You flip an unbiased coin (heads are as likely as tails) 10 times. What is the probability that you get as many tails as heads? What is the probability that you get all heads?
}

Denote the number of heads as $X$. Then $X \sim Bin(10, 0.5)$. We want to find $P(X = 5)$. We calculate this as follows:
\begin{align*}
    P(X = k) &= \frac{{n \choose k}}{2^n} \\
    P(X = 5) &= \frac{{10 \choose 5}}{2^{10}} \\
    &= \frac{\frac{10!}{5!\cdot(10-5)!}}{2^10}\\
    &= \frac{252}{1024} = 0.25\\
    &= 25\%
\end{align*}

Do the same for $P(X = 10)$:
\begin{align*}
    P(X = 10) &= \frac{{10 \choose 10}}{2^{10}} \\
    &= \frac{\frac{10!}{10!\cdot(10-10)!}}{2^10}\\
    &= \frac{1}{1024} = 0.00098\\
    &= 0.098\%
\end{align*}

\subsection{Q2}
\question{
  You throw three ordinary 6-sided unbiased dice. What is the expected sum of the dice? What is the expected product of the dice? 
}

For the sum we can simply use the linearity of expectation, such that the expected value of one die is:
\begin{align*}
    E(X) &= \sum_{i=1}^6 i \cdot P(X = i) \\
    &= (1 \cdot \frac{1}{6}) + (2 \cdot \frac{1}{6}) + (3 \cdot \frac{1}{6}) + (4 \cdot \frac{1}{6}) + (5 \cdot \frac{1}{6}) + (6 \cdot \frac{1}{6})\\
    &= \frac{7}{2} = 3.5
\end{align*}

Then we can simply multiply this by 3 to get the expected sum of the dice:
\begin{align*}
    E(X) &= 3 \cdot E(X) \\
    &= 3 \cdot 3.5 \\
    &= 10.5
\end{align*}

We cannot use the same method for the product but can instead use the definition of expectation, where we sum over all possible values of the random variable. If we consider this for a moment, that means taking every possible die value, 3 times, and multiplying them together. This is a bit more complicated, but we can still do it:
\begin{align*}
    E(X) &= \sum_{x\in \left\{1,2,3,4,5,6\right\}^3} \frac{1}{6^3} \prod_{i=1}^3 x_i \\
    \intertext{Move the terms around}
    &= \frac{1}{6^3} \prod_{i=1}^3\sum_{x_i=1}^6  x_i \\
    \intertext{Use the known sum of the die}
    &= \frac{1}{6^3} \prod{i=1}^3 21 \\
    &= \left(\frac{21}{6}\right)^3 \\
    &= \left(\frac{7}{2}\right)^3 = 42.86
\end{align*}

\subsection{Q3}
\question{
  Suppose you want to sample a random permutation of the first $n$ positive integers uniformly (meaning that all of the $n!$ permutations are equally likely to be the outcome). How can you use an unbiased coin to sample such a permutation. How many coin flips do you need? 
}

We can use the following algorithm to sample a random permutation of the first $n$ positive integers:
\begin{algorithm}[H]
  \caption{Random permutation of the first $n$ positive integers}
  \begin{algorithmic}[1]
    \State Initialize an empty list $L$.
    \For{$i$ from $1$ to $n$}
      \State Flip a coin.
      \If{heads}
        \State Append $i$ to the end of $L$.
      \Else
        \State Insert $i$ at the beginning of $L$.
      \EndIf
    \EndFor
    \State \Return $L$.
  \end{algorithmic}
\end{algorithm}

This algorithm will generate a random permutation of the first $n$ positive integers. The number of coin flips needed is $n$.

\subsection{Q4}
\question{
  We used the bound $(1 + x)^t \leq e^{xt}$ in the lecture. Prove that $(1 + x)^t \leq e^{xt}$ for all $x \geq -1$ and positive $t$ with equality if and only if $x = 0$. Hint: use the Maclaurin expansion $e^x = \sum^\infty_{i=0} \frac{x^i}{i!}$.
}

If we insert $x=0$ then we always get equality since $e^0 = 1$ and $1^t=1$. We can then consider the case where $x > 0$ and $x < 0$ separately.

Using the Maclaurin expansion of $e^x$ we get:
\begin{align*}
  e^x &= \sum^\infty_{i=0} \frac{x^i}{i!} \\
  \intertext{This series looks like:}
  &= 1 + x + \frac{x^2}{2} + \frac{x^3}{6} + \ldots
  \intertext{So we can rewrite it such that}
  e^x &= 1+x+\sum^\infty_{i=2} \frac{x^i}{i!} \\
  \intertext{So we must prove that the sum expression is greater than or equal to 0}
  \intertext{We can rewrite the sum further as}
  e^{x} &= 1 + x + \sum^\infty_{i=1} \left(\frac{x^{2i}}{(2i)!}+\frac{x^{2i+1}}{(2i+1)!}\right) \\
  \intertext{Each of these terms much logically be greater than 0 since $x$ is positive}
  \intertext{In the negative case the first term is positive and much larger in magnitude than the other (negative) one}
\end{align*}

Since this has been proven to be always be positive then it follows that $(1 + x)^t \leq e^{xt}$ since both sides are non negative and that $e^{xt}$ is logically larger than $(1 + x)^t$.

\subsection{Q5}
\question{
  Suppose you have a biased coin that gives heads with some unknown probability $p \neq .5$, but $0 < p < 1$. How can you use the biased coin to simulate an unbiased coin toss, i.e., making the probabilities of the simulated outcomes heads and tails equal?
}

We simply flip the coin twice. If we get $H, T$ we output $H$ and if we get $T,H$ we output $T$. In any other case we keep flipping the coin until we get one of these cases.

The probability of getting $H,T$ or $T,H$ is the same due to the independency of trials, no matter how biased $P$ is. The number of required coin flips largely depends on the bias of the coin, and if the bias is close to 0 or 1, then the number of required flips will be extremely high. But the closer to 0.5 the bias is, the fewer flips are required.
