%!TEX root = ../[BDSA'21] - Exam Answers.tex

\begin{itemize}
	\setlength\itemsep{0em}
	\item This exam comprises of \epages pages (including this one).  There are \equestions questions containing exercises (sub-questions). Each exercise is worth the number of points specified at the beginning of the exercise, which sum up to a total of \epoints points.
	\item You must achieve at least \ethreshold of the marks to pass, i.e., \epointstopass points.
	\item You have \elength to complete this exam.
	\item You must complete the \textbf{Statement of Authorship}.
	\item You must use the \LaTeX\ skeleton provided to you for your final answers (note: these provide you also with hints for the answers).  You must \textbf{only submit the resulting pdf document}. 
	\item Additional three pages of details on how to use the \LaTeX\ skeleton are provided in Appendix \ref{sec:skeleton_instructions}.
	\end{itemize}

\vfill
\begin{itemize}
	\setlength\itemsep{0em}
	\item You have been given ample time to familiarize with the \LaTeX\ skeleton, and it is expected that you use it as designed.
	\item If you follow the \LaTeX\ skeleton, you will be required to modify and include files only in the \el{./solutions} folder.  All macros for the answers have been isolated in the \el{2021\_answers.tex} file.  The folder also contains the \el{./code} and \el{./diagrams} folder in which you are expected to place code and diagrams respectively.
	\item Your program code does not need to be 100\% syntactically correct.  However, given that you are allowed to use an IDE, the verification of the syntax will be less lenient.
	\item Your diagrams, do not need to be drawn using a specific software, and you are welcomed to draw it on paper, take a picture, and use the hand drawn picture as diagram.  Make sure that the contrast allows to clearly understand the diagram.  Macros have been tested with .png and .pdf files.
\end{itemize}



