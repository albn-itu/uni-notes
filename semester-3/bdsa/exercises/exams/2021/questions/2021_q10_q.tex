%!TEX root = ../[BDSA'21] - Exam Answers.tex

\pgfmathsetmacro\totalpoints{\totalpoints + 11}

\noindent You have a method with the following signature:
\begin{lstlisting}
public class DoubleO
{
    public static string GetAgent(int number) { ... }
}
\end{lstlisting}
\noindent You have the following requirements for input and expected return:
\begin{center}
\begin{tabular}{ c l }
    \# & Name \\
    \hline
    006 & Alec Trevelyan \\
    007 & James Bond \\
    000 & \texttt{ArgumentOutOfRangeException} \\
    \hline
\end{tabular}
\end{center}





\vskip 15pt
\begin{enumerate}[a]
    \item \point{5} Implement a set of tests for the \texttt{GetAgent} method:
		\input{\questionTenAnswerA}
\def\questionTenAnswerB{answers/code/q10_answerB} %% edit the content of the file
\def\questionTenAnswerC{answers/code/q10_answerC} %% edit the content of the file





\newpage
\noindent Consider the following interface:
\begin{lstlisting}
public interface IQService
{
    Gadget GetRandom();
}
\end{lstlisting}
\noindent A gadget should have a name and a film.




\vskip 15pt
    \item \point{2} Implement the \texttt{Gadget} class such that name and film are settable only using an object initializer and ensure that none of them will be \texttt{null}.
		\input{\questionTenAnswerB}
\def\questionTenAnswerC{answers/code/q10_answerC} %% edit the content of the file




\noindent You need to use the \texttt{IQService} in the context of a unit test.\\
\noindent You need to setup a fake implementation which returns ``Milk bottle grenades'' from ``The Living Daylights''.\\
\noindent You should set the value of \texttt{service} to be the runtime implementation of your fake.





\vskip 15pt
    \item \point{4} Implement the fake:
		\input{\questionTenAnswerC}

\end{enumerate}
