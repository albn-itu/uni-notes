%!TEX root = ../[BDSA'21] - Exam Answers.tex

\pgfmathsetmacro\totalpoints{\totalpoints + 14}

Given the following functional requirements:

\begin{itemize}
    \item FR01: A \texttt{Semester} has a numeric \texttt{Id}, a numeric \texttt{Year}, and a \texttt{Label} of either \texttt{Fall} or \texttt{Spring}.
    \item FR02: A \texttt{Course} has a numeric \texttt{Id}, a \texttt{Title}, a \texttt{Semester}, and a list of \texttt{Teacher}s.
    \item FR03: A \texttt{Teacher} has a \texttt{Title}, a \texttt{Name}, and an \texttt{Email}, and a list of \texttt{Course}s.
    \item FR04: Only valid email addresses may be stored in the system.
    \item FR05: The following values are \textit{always} needed: \texttt{Year}, \texttt{Label}, \texttt{Title}, and \texttt{Name}.
    \item FR06: All text values must be at most 50 characters long.
    \item FR07: A \texttt{Course} may be stored without knowing the \texttt{Semester}.
\end{itemize}

\vskip 15pt
\begin{enumerate}[a]
    \item \point{5} Draw a UML diagram which illustrates the functional requirements with relations between the entities.

\fullline\vspace{-8pt}
\begin{center}{
	\scriptsize{\emph{Use this space between the horizontal lines to draw your final UML diagram.}}
	\includegraphics[width=\questionTwoAnswerAwidth\columnwidth]{\questionTwoAnswerA}
	}
\end{center}
\vfill
\fullline





\vskip 15pt
    \item \point{5} Implement a model which allows you to store the model in a relational database using Entity Framework Core.
		\input{\questionTwoAnswerB}





\vskip 15pt
    \item \point{2} Implement the code which creates the course \textit{BDSA} of \textit{Fall 2021} given by \textit{Prof.} \text{Lystrøm} and \textit{Prof.} \textit{Tell} (email unknown).
		\input{\questionTwoAnswerC}





\newpage
    \item \point{2} Implement the code which retrieves all \textit{Spring} semester courses teached by \textit{Tell} projecting an anonymous type of \textit{Title} and \textit{Year}.
		\input{\questionTwoAnswerD}

\end{enumerate}




